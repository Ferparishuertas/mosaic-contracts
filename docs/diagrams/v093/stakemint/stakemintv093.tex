\documentclass[landscape, 11pt, svgnames]{article}

\usepackage[showframe]{geometry}
\usepackage[T1]{fontenc}
\usepackage[utf8]{inputenc}
\usepackage[english]{babel}
\usepackage{listings}
\usepackage{tikz-uml}
\usepackage{amsfonts, amsmath, amsthm, amssymb}

\geometry{
  paperwidth=65cm,
  paperheight=112cm,
  margin=1cm
}

\date{\today}
\title{OpenST Protocol sequence diagrams v0.9.3}
\author{Benjamin Bollen}

\lstdefinelanguage{tikzuml}{language=[LaTeX]TeX, classoffset=0, morekeywords={umlbasiccomponent, umlprovidedinterface, umlrequiredinterface, umldelegatewayconnector, umlassemblyconnector, umlVHVassemblyconnector, umlHVHassemblyconnector, umlnote, umlusecase, umlactor, umlinherit, umlassoc, umlVHextend, umlinclude, umlstateinitial, umlbasicstate, umltrans, umlstatefinal, umlVHtrans, umlHVtrans, umldatabase, umlmulti, umlobject, umlfpart, umlcreatecall, umlclass, umlvirt, umlunicompo, umlimport, umlaggreg}, keywordstyle=\color{DarkBlue}, classoffset=1, morekeywords={umlcomponent, umlsystem, umlstate, umlseqdiag, umlcall, umlcallself, umlfragment, umlpackage}, keywordstyle=\color{DarkRed}, classoffset=0,  sensitive=true, morecomment=[l]{\%}}

\begin{document}
\begin{minipage}[b]{0.55\linewidth}
\Huge \color{NavyBlue} \textbf{OpenST Protocol v0.9.3 } \color{Black}\\ % Title
\huge\textit{sequence diagrams for stake and mint - Benjamin Bollen \& Pranay Valson, last edit \today}\\[1cm] % Subtitle
\end{minipage}

\begin{tikzpicture}
  \begin{umlseqdiag}
    \umlactor[class=Address]{Staker}
    \umlactor[class=Worker]{Facilitator}
    \umlactor[class=Worker]{Hunter}
    \umlboundary[class=ERC20]{OST} 
    \umlboundary[class=SK]{Branded Token Gateway}
    \umlcontrol[class=SK]{OpenSTValue}
    \umlobject[class=SK]{SimpleStake}
    \umlobject[class=SK]{CoreUC}
    %\umlcontrol[class=SK]{RegistrarVC}
    \umlboundary[class=Web3]{Value Chain}
    \umlactor[class=Worker, fill=purple!40]{OstDotCom}
    \umlboundary[class=Web3, fill=blue!40]{Utility Chain}
    %\umlcontrol[class=SK, fill=blue!40]{RegistrarUC}
    \umlobject[class=SK, fill=blue!40]{CoreVC}
    \umlcontrol[class=SK, fill=blue!40]{OpenSTUtility}
    \umlboundary[class=ERC20, fill=blue!40]{Branded Token}
   \umlactor[class=Address, fill=blue!40]{Beneficiary} %Token Holder contract in v0.9.4
  
    %%%
    %%%  Staker initiates stake
    %%% 
    
    % staker approves branded token gateway contract on OST for amount
    \begin{umlcall}[op={: approve(gateway, amountST)}]{Staker}{OST}  
    \end{umlcall}
    % worker approves worker contract for bounty amount
   % \begin{umlcall}[dt=9,op={: approve(worker, bounty)}]{Facilitator}{Workers}  
   % \end{umlcall}
          
    % staker requests stake to gateway
    \begin{umlcall}[dt=10, op={: requestStake(amountST, beneficiary)}]{Staker}{Branded Token Gateway}
      % gateway pulls amount from staker on OST
      \begin{umlcall}[fill=green!20, dt=5, op={: transferFrom(staker, gateway, amountST)}, return=<<OST(amountST)>>]{Branded Token Gateway}{OST}
        % pull OST from Facilitator
        \begin{umlcall}[dt=5, fill=green!20, type=return, op={<<OST(amountST)>>}]{Staker}{OST}
        \end{umlcall}
      \end{umlcall}
      
      % emit StakeRequested event which Facilitator listens to
      \begin{umlcall}[dt=5, type=return, op={emit StakeRequested(staker, amountST, beneficiary)}]{Branded Token Gateway}{Facilitator}
      \end{umlcall}
    \end{umlcall}
   


    % Facilitator evaluates request against policy (monetary and KYC/AML)
    \begin{umlfragment}[type=alt, label=accept, name=policy, inner xsep=2]
      \begin{umlcall}[dt=10, op={: approve(gateway, bounty)}]{Facilitator}{OST}
      \end{umlcall}
      % Facilitator accepts request
      \begin{umlcall}[op={: acceptStakeRequest(staker, hashLock)}]{Facilitator}{Branded Token Gateway}  %return={ emit StakingRequestAccpeted (staker, amountST, amountUT, nonce, unlockHeight, stakingIntentHash}
        % Pull Facilitator's bounty into Gate
        \begin{umlcall}[fill=green!20, op={: transferFrom(facilitator, gateway, bounty)}, return={<<OST(bounty)>>}]{Branded Token Gateway}{OST}
            % pull OST for bounty from Facilitator
            \begin{umlcall}[dt=5, fill=green!20, op={<<OST(bounty)>>},type=return]{Facilitator}{OST}
            \end{umlcall}
        \end{umlcall}
        % Approve OpenSTValue as spender for Gate
        \begin{umlcall}[op={: approve(openSTValue, amountST)}]{Branded Token Gateway}{OST}
        \end{umlcall}
        
        % Gate calls on to OpenSTValue to stake (later abstract to library call)
        \begin{umlcall}[op={: stake(uuid, amountST, beneficiary, hashLock, staker)},return={amountUT, nonce, unlockHeight, StakingIntentHash}]{Branded Token Gateway}{OpenSTValue} % 
          % check stakingAccount 
          \begin{umlcallself}[op={require(msg.sender = gateway = stakingAccount || stakingAccount = 0x)}]{OpenSTValue}
          \end{umlcallself}
          % OpenSTValue pulls amount plus bounty from Facilitator to
          % its OST account balance
          \begin{umlcall}[dt=4, op={: transferFrom(gateway, OpenSTValue, amountST)}, fill=green!20, return=<<OST(amountST)>>]{OpenSTValue}{OST}
            % pull OST for pre-fund amount from Facilitator
            \begin{umlcall}[dt=20, fill=green!20, type ={return},op ={<<OST(amountST)>>}]{Branded Token Gateway}{OST}
            \end{umlcall}
          \end{umlcall}
          % store StakingIntentHash in contract storage
          \begin{umlcallself}[op={store StakingIntentHash},]{OpenSTValue}
          \end{umlcallself}
          % HTLC(facilitator, amount+bounty)Facilitator
         % \begin{umlcallself}[dt=0, op={HTLC(staker, amount)},]{OpenSTValue}
          %\end{umlcallself}
          % emit StakingIntentDeclared
          \begin{umlcall}[type=return, op={emit StakingIntentDeclared(uuid, staker, nonce, intentKeyHash, beneficiary, amountST, amountUT, unlockHeight, StakingIntentHash, chainId)}]{OpenSTValue}{Facilitator}
          \end{umlcall}
        \end{umlcall}
        % HTLC(facilitator, bounty)
        %\begin{umlcallself}[op={HTLC(facilitator, bounty)}]{Branded Token Gateway}
        %\end{umlcallself}
      \end{umlcall}
      
      % Facilitator rejects request
      \umlfpart[reject]
      \begin{umlcall}[dt=5, op={: rejectStakeRequest(staker, reason)}]{Facilitator}{Branded Token Gateway}
        % gateway checks msg.sender == whitelister worker address and there is no hashlock
        \begin{umlcallself}[op={require(msg.sender=worker \&\& no HTLC)}]{Branded Token Gateway}
        \end{umlcallself}      
        % transfer amount back to staker
        \begin{umlcall}[fill=green!20, op={: transfer(staker, amountST)}]{Branded Token Gateway}{OST}
          % return OST to staker
          \begin{umlcall}[type=return, fill=green!20, op=<<OST(amountST)>>]{OST}{Staker}
          % emit stake request rejected 
          \begin{umlcall}[dt=10, type=return, op={emit StakeRequestRejected(staker, amountST, reason)}]{Branded Token Gateway}{Facilitator}
          \end{umlcall}
        \end{umlcall}
      \end{umlcall}
      \end{umlcall}
    \end{umlfragment}
    \umlnote[x=2, y=-7]{policy}{evaluate request against policy (KYC/AML)}
    \umlnote[x=30,y=-5, width=200]{policy}{to accept the staking request the facilitator generates a secret, random unlockSecret, and publishes hashLock=Hash(unlockSecret)}
    
    % optionally, staker can initiate revert after timeout and no action from Facilitator
    \begin{umlfragment}[type=opt]
      % staker reverts Stake Request after timeout
      \begin{umlcall}[dt=12, op={: revertStakeRequest()}, fill=green!20]{Staker}{Branded Token Gateway}
        % gateway checks amount is not locked under HTLC or time-lock is not yet expired
        \begin{umlcallself}[op={require(msg.sender=staker \&\& no HTLC)}]{Branded Token Gateway}
        \end{umlcallself}
        % transfer amount back to staker
        \begin{umlcall}[fill=green!20, op={: transfer(staker, amountST)}]{Branded Token Gateway}{OST}
          % return OST to staker
          \begin{umlcall}[type=return, fill=green!20, op=<<OST(amountST)>>]{OST}{Staker}
          \end{umlcall}
          % emit rejectStakeRequest event
          \begin{umlcall}[type=return, op={: emit StakeRequestReverted(staker, OST(amountST))}]{OST}{Facilitator}
          \end{umlcall}          
        \end{umlcall}
      \end{umlcall}
    \end{umlfragment}
    
    %%%
    %%%  OpenST Mosaic (OstDotCom reports respective state root of )
    %%%
    
    \begin{umlfragment}[type=loop, name=mosaic]
      
      \begin{umlcall}[dt=198, op={new block(blockHeightUC, stateRootUC)}]{Utility Chain}{OstDotCom}
        % OstDotCom report state root of Utility chain on CoreUC on value chain
        %\begin{umlcall}[op={: report(UC, blockHeight, stateRoot)}]{OstDotCom}{RegistrarVC}
          \begin{umlcall}[op={: commitStateRoot(blockHeightUC, stateRootUC)}]{OstDotCom}{CoreUC}
          % make sure block height is greater than the lastestStateRootBlockHeight previously stored
          \begin{umlcallself}[op={require(blockHeightUC > latestSateRootBlockHeightUC)}]{CoreUC}
          \end{umlcallself}
          % store the state root in the contract's mapping
          \begin{umlcallself}[dt=4,op={store latest valid stateRootUC}]{CoreUC}
          \end{umlcallself}
            % state root of utility chain got reported on value chain
            \begin{umlcall}[dt=10, type=return, op={emit StateRootCommitted(blockHeightUC, stateRootUC)}]{CoreUC}{Facilitator}
            \end{umlcall}
          \end{umlcall}
        %\end{umlcall}
      \end{umlcall}
          
      % OstDotCom report state root of Value chain on CoreVC on utility chain
      \begin{umlcall}[dt=212, op={new block(blockHeightVC, stateRootVC)}]{Value Chain}{OstDotCom}
        %\begin{umlcall}[op={: report(VC, blockHeight, stateRoot}]{OstDotCom}{RegistrarUC}
          % reporting through registrar is instant commit
          \begin{umlcall}[op={: commitStateRoot(blockHeightVC, stateRootVC)}]{OstDotCom}{CoreVC}
          % make sure block height is greater than the lastestStateRootBlockHeight previously stored
          \begin{umlcallself}[op={require(blockHeightVC > latestSateRootBlockHeightVC)}]{CoreVC}
          \end{umlcallself}
          % store the state root in the contract's mapping
          \begin{umlcallself}[op={store latest valid stateRootVC}]{CoreVC}
          \end{umlcallself}          
            % state root of value chain got reported on utility chain
            \begin{umlcall}[dt=10, type=return, op={emit StateRootCommitted(blockHeightVC, stateRootVC)}]{CoreVC}{Facilitator}
            \end{umlcall}
          \end{umlcall}
        %\end{umlcall}
      \end{umlcall}
       \end{umlfragment}
    
    %%%
    %%% Facilitator has observed a committed state root that includes the StakingIntentHash
    %%%
    
    
       % Facilitator submits claim for StakingIntentHash by presenting Merkle proof
    \begin{umlcall}[dt=15, op={: proveOpenST(blockHeightVC, rlpEncodedAccount, rlpParentNodes)}, return={emit OpenSTProven(blockHeightVC, storageRootVC, hashedAccount)}]{Facilitator}{CoreVC}
    % OpenSTUtility checks StakingIntentHash against committed state root
     % \begin{umlcall}[op={: getStateRoot(blockHeight)}, return={stateRoot @ blockHeight}]{OpenSTUtility}{CoreVC}
     %\end{umlcall}
      % OpenSTUtility validate merkle proof
      \begin{umlcallself}[op={require(verify(hashedAccount, encodedOpenSTRemotePath, rlpParentNodes, stateRootVC))}]{CoreVC}
     \end{umlcallself}
      % OpenSTUtility store valid StakingIntentHash
      \begin{umlcallself}[op={store latest valid storageRootVC}]{CoreVC}
      \end{umlcallself}
    \end{umlcall}

  
      \umlnote[x=31, y=-28, width=180]{mosaic}{Placeholder for OpenST Mosaic game}
    % Facilitator submits pre-image data for StakingIntentHash
    \begin{umlcall}[dt=15, op={: confirmStakingIntent(uuid, staker, stakerNonce, beneficiary, amountST, amountUT, stakingUnlockHeight, hashLock, rlpParentNodes)}, return={emit StakingIntentConfirmed(stakingIntentHash, staker, beneficiary, amountST, amountUT, expirationHeight, blockHeight, storageRootVC)}]{Facilitator}{OpenSTUtility}
	% check stake nonces of staker is less than staker nonce      
     % \begin{umlcallself}[op={require(nonces[staker] < stakerNonce)}]{OpenSTUtility}
     % \end{umlcallself} 
     % check stakingUnlockHeight is greater than 0
      \begin{umlcallself}[op={require(stakingUnlockHeight > 0)}]{OpenSTUtility}
      \end{umlcallself} 
     % calculate the stakingIntentHash
      %\begin{umlcallself}[op={stakingIntentHash = H(uuid, staker, stakerNonce, beneficiary, amountST, amountUT, stakingUnlockHeight, hashLock)}]{OpenSTUtility}
      %\end{umlcallself}         
     % check calculated stakingIntentHash is equal to provided stakingIntentHash
      \begin{umlcallself}[op={require(merkleVerificationOfStake(StakingIntentHash, rlpParentNodes, storageRoot)}]{OpenSTUtility}
      \end{umlcallself}       
      %store the mints struct
      \begin{umlcallself}[op={store mints[stakingIntentHash]}]{OpenSTUtility}
      \end{umlcallself}
      % OpenSTUtility asserts valid pre-image data for StakingIntentHash
      %\begin{umlcallself}[op={assert valid pre-image data}]{OpenSTUtility}
      %\end{umlcallself}
      % OpenSTUtility checks StakingIntentHash against committed state root
      %\begin{umlcall}[op={: getLatestHeight()}, return={latestHeight}]{OpenSTUtility}{CoreVC}
      %\end{umlcall}
      % OpenSTUtility asserts grace period before unlockHeight
      %\begin{umlcallself}[op={assert grace period before unlockHeight}]{OpenSTUtility}
      %\end{umlcallself}
      % OpenSTUtility store mint object with StakingIntentHash and expiration Height
      %\begin{umlcallself}[op={store mint with expirationHeight}]{OpenSTUtility}
      %\end{umlcallself}
    \end{umlcall}
    
    %%%
    %%% Facilitator has moved both value and utility chain to the first stage
    %%% and can now either proceed or revert by revealing the hash lock secret or await timeout
    %%%
    
    % Correctly initialised staking information on both systems
    \begin{umlfragment}[type=alt, label=proceed, name=phasetwo, inner xsep=1.5]
      \begin{umlcall}[dt=8, fill=green!20, op={: processStaking(StakingIntentHash, unlockSecret)}]{Facilitator}{Branded Token Gateway}
        % check hashlock 
        \begin{umlcallself}[op={require(H(unlockSecret) = hashLock)}]{Branded Token Gateway}
        \end{umlcallself}
        % check msg.sender = worker
        %\begin{umlcallself}[op={require(msg.sender = worker)}]{Branded Token Gateway}
        %\end{umlcallself}        
        % Facilitator first calls processStaking to ensure the bounty is return to him
        \begin{umlcall}[dt=4, op={: processStaking(StakingIntentHash, unlockSecret)}]{Branded Token Gateway}{OpenSTValue}
          % check stakingAccount 
          %\begin{umlcallself}[op={require(msg.sender = gateway = stakingAccount || stakingAccount = 0x)}]{OpenSTValue}
          %\end{umlcallself}
          % check hashlock 
          \begin{umlcallself}[op={require(H(unlockSecret) = hashLock)}]{OpenSTValue}
          \end{umlcallself}
          % transfer amount OST to SimpleStake
          \begin{umlcall}[fill=green!20, op={: transfer(simpleStake, amountST)}]{OpenSTValue}{OST}
            \begin{umlcall}[dt=5, fill=green!20, type=return, op={<<OST(amountST)>>}]{OST}{SimpleStake}
            \end{umlcall}
          \end{umlcall}
          \begin{umlcall}[dt=5, type=return, op={emit ProcessedStake(StakingIntentHash, staker, amountST, amountUT, unlockSecret)}]{OpenSTValue}{Hunter}
          \end{umlcall}
        \end{umlcall}
        % transfer bounty from gateway to facilitator
        \begin{umlcall}[dt=5, fill=green!20, op={: transfer(facilitator, bounty)}]{Branded Token Gateway}{OST}
          \begin{umlcall}[fill=green!20, type=return, op={<<OST(bounty)>>}]{OST}{Facilitator}
          % \begin{umlcall}[fill=green!20, type=return, op={<<OST(bounty)>>}]{Workers}{Facilitator}
          %\end{umlcall}
          \end{umlcall}
        \end{umlcall}
      \end{umlcall}
      
      % Facilitator then calls processMinting to mint the utility tokens
      \begin{umlcall}[op={: processMinting(StakingIntentHash, unlockSecret)}]{Facilitator}{OpenSTUtility}
        % check hashlock 
        \begin{umlcallself}[op={require(H(unlockSecret) = hashLock)}]{OpenSTUtility}
        \end{umlcallself}
        % require mint not yet expired
        \begin{umlcallself}[op={require(expirationHeight > block.number)}]{OpenSTUtility}
        \end{umlcallself}
        % mint Branded Tokens
        \begin{umlcall}[fill=green!20, op={: mint(beneficiary, amountUT)}]{OpenSTUtility}{Branded Token}
          % store claim
          \begin{umlcallself}[op={store claim}]{Branded Token}
          \end{umlcallself}
        \end{umlcall}
        \begin{umlcall}[type=return, op={emit ProcessedMint(StakingIntentHash, tokenAddress, staker, beneficiary, amountUT, unlockSecret)}]{OpenSTUtility}{Hunter}
        \end{umlcall}
      \end{umlcall}
      
      % Facilitator (or anyone) can call claim to transfer UT to token holder
      \begin{umlcall}[dt=5, op={: claim(beneficiary)}]{Facilitator}{Branded Token}
        \begin{umlcall}[fill=green!20, type=return, op={<<UT(amountUT)>>}]{Branded Token}{Beneficiary}
        \end{umlcall}
      \end{umlcall}
      
      %%%
      %%%  Hunter
      %%%
      
      \begin{umlfragment}[type=opt, name=bounty, label={ensure completion}, inner xsep=15]
        % process staking on gateway
        \begin{umlcall}[dt=20, fill=green!20, op={: processStaking(StakingIntentHash, unlockSecret)}]{Hunter}{Branded Token Gateway}
        % check hashlock 
        \begin{umlcallself}[op={require(H(unlockSecret = hashLock)}]{Branded Token Gateway}
        \end{umlcallself}
        % check msg.sender = worker
       % \begin{umlcallself}[op={require(msg.sender = worker)}]{Branded Token Gateway}
        %\end{umlcallself}        
          % Facilitator first calls processStaking to ensure the bounty is return to him
          \begin{umlcall}[dt=4, op={: processStaking(StakingIntentHash, unlockSecret)}]{Branded Token Gateway}{OpenSTValue}
            % check stakingAccount 
           % \begin{umlcallself}[op={require(msg.sender = gateway = stakingAccount || stakingAccount = 0x)}]{OpenSTValue}
            %\end{umlcallself}
            % check hashlock 
            \begin{umlcallself}[op={require(H(unlockSecret) = hashLock)}]{OpenSTValue}
            \end{umlcallself}
            % transfer amount OST to SimpleStake
            \begin{umlcall}[fill=green!20, op={: transfer(simpleStake, amountST)}]{OpenSTValue}{OST}
              \begin{umlcall}[dt=5, fill=green!20, type=return, op={<<OST(amountST)>>}]{OST}{SimpleStake}
              \end{umlcall}
            \end{umlcall}
            \begin{umlcall}[dt=5, type=return, op={emit ProcessedStake(StakingIntentHash, simpleStake, staker, amountST, amountUT, unlockSecret)}]{OpenSTValue}{Facilitator}
            \end{umlcall}
          \end{umlcall}
          % transfer bounty from gateway to facilitator
          \begin{umlcall}[dt=5, fill=green!20, op={: transfer(hunter, bounty)}]{Branded Token Gateway}{OST}
            \begin{umlcall}[fill=green!20, type=return, op={<<OST(bounty)>>}]{OST}{Hunter}
            \end{umlcall}
          \end{umlcall}
        \end{umlcall}
      \end{umlfragment}
      \umlnote[x=12, y=-66, width=150]{bounty}{If stake was left unprocessed, unlock secret is known through mint. Bounty is always transferred to msg.sender of processStaking()}

      \umlfpart[revert]
     
      \begin{umlcall}[dt=20, fill=green!20, op={: revertStaking(StakingIntentHash)}]{Facilitator}{Branded Token Gateway}
        % check unlockHeight is in the past
        \begin{umlcallself}[op={require(unlockHeight <= block.number)}]{Branded Token Gateway}
        \end{umlcallself}
          % check msg.sender = worker
          %\begin{umlcallself}[op={require(msg.sender = worker)}]{Branded Token Gateway}
          %\end{umlcallself}        
        % Facilitator reverts stake to get back amount and bounty
        \begin{umlcall}[fill=green!20, op={: revertStaking(StakingIntentHash)}]{Branded Token Gateway}{OpenSTValue}
          % check stakingAccount 
         % \begin{umlcallself}[op={require(msg.sender = gateway = stakingAccount || stakingAccount = 0x)}]{OpenSTValue}
          %\end{umlcallself}
          % check unlockHeight
          \begin{umlcallself}[op={require(unlockHeight <= block.number)}]{OpenSTValue}
          \end{umlcallself}
          % transfer amount and bounty OST to facilitator
          \begin{umlcall}[fill=green!20, op={: transfer(staker, amountST)}]{OpenSTValue}{OST}
            \begin{umlcall}[fill=green!20, type=return, op={<<OST(amountST)>>}]{OST}{Staker}
            \end{umlcall}
           \begin{umlcall}[dt=10, type=return, op={emit RevertedStake(StakingIntentHash, staker, amountST, amountUT)}]{OpenSTValue}{Facilitator}
            \end{umlcall}   
          \end{umlcall}
        \end{umlcall}
        % transfer bounty from gateway to facilitator
        \begin{umlcall}[dt=5, fill=green!20, op={: transfer(facilitator, bounty)}]{Branded Token Gateway}{OST}
          \begin{umlcall}[fill=green!20, type=return, op={<<OST(bounty)>>}]{OST}{Facilitator}
          %\begin{umlcall}[fill=green!20, type=return, op={<<OST(bounty)>>}]{Workers}{Facilitator}
          %\end{umlcall}
          \end{umlcall}
        \end{umlcall}
      \end{umlcall} 

      \begin{umlcall}[op={: revertMinting(StakingIntentHash)}]{Facilitator}{OpenSTUtility}
        % require mint not yet expired
        \begin{umlcallself}[op={require(expirationHeight <= block.number)}]{OpenSTUtility}
                \end{umlcallself}
           \begin{umlcallself}[op={delete mints[StakingIntentHash]}]{OpenSTUtility}
                \end{umlcallself}
          \begin{umlcall}[dt=5, type=return, op={emit RevertedMint(uuid, staker, beneficiary, amountUT)}]{OpenSTUtility}{Facilitator}
          \end{umlcall}

      \end{umlcall}
    \end{umlfragment}
    \umlnote[x=1, y=-55, width=100]{phasetwo}{With both value chain and utility chain configured correctly for StakingIntentHash, Facilitator can proceed by revealing the unlock secret, or revert by awaiting the unlock height }
    \umlnote[x=35, y=-102, width=120]{phasetwo}{Any actor can call revertStaking and the bounty is returned to the facilitator}

  \end{umlseqdiag}
\end{tikzpicture}

\begin{minipage}[b]{0.55\linewidth}
\Huge \color{NavyBlue} \textbf{OpenST Protocol v0.9.3 } \color{Black}\\ % Title
\huge\textit{sequence diagrams for redeem and unstake - Benjamin Bollen \& Pranay Valson, last edit \today}\\[1cm] % Subtitle
\end{minipage}

\begin{tikzpicture}
  \begin{umlseqdiag}
     \umlactor[class=Address]{Beneficiary}
   \umlactor[class=Worker]{Facilitator}
   \umlactor[class=Worker]{Hunter}
    \umlboundary[class=ERC20]{OST} 
    \umlboundary[class=SK]{Branded Token Gateway}
    \umlcontrol[class=SK]{OpenSTValue}
    \umlobject[class=SK]{SimpleStake}
    \umlobject[class=SK]{CoreUC}
    %\umlcontrol[class=SK]{RegistrarVC}
    \umlboundary[class=Web3]{Value Chain}
    \umlactor[class=Worker, fill=purple!40]{OstDotCom}
    \umlboundary[class=Web3, fill=blue!40]{Utility Chain}
    %\umlcontrol[class=SK, fill=blue!40]{RegistrarUC}
    \umlobject[class=SK, fill=blue!40]{CoreVC}
    \umlcontrol[class=SK, fill=blue!40]{OpenSTUtility}
    \umlboundary[class=ERC20, fill=blue!40]{Branded Token}
   \umlactor[class=Address, fill=blue!40]{Redeemer} %Token Holder contract in v0.9.4
  
   %%%
    %%%  redeemer initiates redeem
    %%% 
    %start with redeem process on the VC side
    % redeemer sets allowance for branded token with OpenSTUtility
    \begin{umlcall}[dt =7, op={: approve(OpenSTUtility, amountUT)}]{Redeemer}{Branded Token}  
    \end{umlcall}
    % redeemer calls redeem on OpenSTUtility 
    
   \begin{umlfragment}[type=alt, label=redeem, name=policy]
    \begin{umlcall}[dt=10, op={: redeem(uuid, amountUT, nonce, beneficiary, hashLock)}]{Redeemer}{OpenSTUtility}
    %require that  allowance is there for OSTU for amount UT
    \begin{umlcallself}[op={require (allowance(redeemer, OpenSTUtility) >= amountUT)}]{OpenSTUtility}
    \end{umlcallself}

    % transferfrom is called from openstutility on branded token
   \begin{umlcall}[fill=green!20,op={: transferFrom(redeemer, OpenSTUtility, amountUT)}]{OpenSTUtility}{Branded Token}
   % ut is taken from redeemer to branded token
   \begin{umlcall}[dt=15,fill=green!20, type=return, op={<<UT(amountUT)>>}]{Redeemer}{Branded Token}
     \end{umlcall}
   % ut is the then transfferred to openstutility from branded token (which keeps balance)
   \begin{umlcall}[dt=14,fill=green!20,  type=return, op={<<UT(amountUT)>>}]{Branded Token}{OpenSTUtility}
       \end{umlcall}
   % store the intents in the storage
   \begin{umlcallself}[dt=15, op={store intents[H(redeemer, nonce)] = redemptionIntentHash}]{OpenSTUtility}
   \end{umlcallself}
   % event is emitted with redemptionintentHash which the worker can listen to
    \begin{umlcall}[type=return, op={emit RedemptionIntentDeclared(uuid, redemptionIntentHash, brandedToken, redeemer, nonce, beneficiary, amountUT, unlockHeight)}]{OpenSTUtility}{OstDotCom}
     \end{umlcall}
    %event can be heard by Redeemer 
    %\begin{umlcall}[type=return, op={emit RedemptionIntentDeclared}]{OpenSTUtility}{Redeemer}
  %\end{umlcall}
  \end{umlcall}   
    \end{umlcall}
     \end{umlfragment}    % Facilitator submits claim for StakingIntentHash by presenting Merkle proof
     
    \begin{umlcall}[dt=90, op={: proveOpenST(blockHeightUC, rlpEncodedAccount, rlpParentNodes)}, return={emit OpenSTProven(blockHeightUC, storageRootUC, hashedAccount)}]{Facilitator}{CoreUC}
    % OpenSTUtility checks StakingIntentHash against committed state root
     % \begin{umlcall}[op={: getStateRoot(blockHeight)}, return={stateRoot @ blockHeight}]{OpenSTUtility}{CoreVC}
     %\end{umlcall}
      % OpenSTUtility validate merkle proof
      \begin{umlcallself}[op={require(verify(hashedAccount, encodedOpenSTRemotePath, rlpParentNodes, stateRootUC))}]{CoreUC}
     \end{umlcallself}
      % OpenSTUtility store valid StakingIntentHash
      \begin{umlcallself}[op={store latest valid storageRootUC}]{CoreUC}
      \end{umlcallself}
    \end{umlcall}
     
     % confirm redemption intent begins on UC side
     % beneficiary on VC side calls confirm redemption intent on openstvalue
    \begin{umlcall}[dt= 45, op={: confirmRedemptionIntent(redeemer, redeemerNonce, beneficiary, amountUT, redemptionUnlockHeight, blockHeight, hashLock, rlpParentNodes)}]{OstDotCom}{OpenSTValue}
    %require redemption unlock height > 0
     \begin{umlcallself}[op={require(expirationHeight <= block.number)}]{OpenSTValue}
       \end{umlcallself}
 
	%require that the redeemer nonce +1 == redeemer nonce passed 
    % \begin{umlcallself}[op={require(nonces[redeemer] +1  = redeemerNonce)}]{OpenSTValue}
	%\end{umlcallself}     
	%calculate amountST from amountUT
     \begin{umlcallself}[op={calculate amountST = (amountUT*conversionRate)}]{OpenSTValue}
	\end{umlcallself}     	
	%make sure there is more balance than what is calculated
     %\begin{umlcallself}[op= {require balanceOf(UT.SimpleStake)>= amountST}]{OpenSTValue}
	%\end{umlcallself}     	
     %require verify  Redemption hash storage
     \begin{umlcallself}[op={require verifyRedemptionIntentHashStorage(uuid, redeemer, redeemerNonce, blockHeight, redemptionIntenHash, rlpParentNodes)}]{OpenSTValue}
     \end{umlcallself}

	% get storage root from UC side
	\begin{umlcall}[op= {: getStorageRoot(blockHeight)}, return = {storageRoot}]{OpenSTValue}{CoreVC}
	\end{umlcall}
	%verify intent storage on VC side
	\begin{umlcallself}[op= {require verifyIntentStorage(redeemer, redeemerNonce,  storageRoot, redemptionIntentHash, rlpParentNodes)}]{OpenSTValue}
	\end{umlcallself}      
	% store the unstakes in the unstakes struct with the redemptionhash as key 
     \begin{umlcallself}[op={store unstakes[redemptionIntentHash]}]{OpenSTValue}
	\end{umlcallself}     
     %emit the remeption Intent confirmed event 
     \begin{umlcall}[type= return, op={emit RedemptionIntentConfirmed(redemptionIntentHash, redeemer, beneificary, amountST, amountUT, expirationHeight )}]{OpenSTValue}{OstDotCom}
      %emit the remeption Intent confirmed event 
     %\begin{umlcall}[type= return]{OpenSTValue}{Utility Chain}
      %\end{umlcall} 
  \end{umlcall}
             \end{umlcall}

 
    \begin{umlfragment}[type=alt, label=proceed, name=policy, inner xsep=2]
	% process redeem should begin now on UC side
	% redeemer calls on process redeeming on openstutility 
    \begin{umlcall}[dt=100, op={: processRedeeming(redemptionIntentHash, unlockSecret)}]{Redeemer}{OpenSTUtility}

    % require hashlock is equal to hash of unlock secret
    \begin{umlcallself}[op={require (hashLock = H(unlockSecret))}]{OpenSTUtility}
    \end{umlcallself}
    
    % burn the UT tokens 
    \begin{umlcall}[fill=green!20, op={: burn(redeemer, amountUT)}]{OpenSTUtility}{Branded Token}
      \end{umlcall}
     %emit the process Redemption event
    \begin{umlcall}[type= return, op={emit ProcessedRedemption(uuid, brandedToken, redeemer, beneficiary, amountUT, unlockSecret)}]{OpenSTUtility}{OstDotCom} 
        \end{umlcall}
    %delete the redemptions struct from storage
    \begin{umlcallself}[op={delete redemptions[redemptionIntentHash])}]{OpenSTUtility}
    \end{umlcallself}
    %delete the redemptions struct from storage
    \begin{umlcallself}[op={delete intents[H(redeemer, nonce)]}]{OpenSTUtility}
    \end{umlcallself}    
  \end{umlcall}



    	%revert unstaking could be initiatied from the redeemer side
	% beneficiary calls on processUnstaking with redempition intent hash
	\begin{umlcall}[dt= 38, op={: processUnstaking(redemptionIntentHash, unlockSecret)}]{OstDotCom}{OpenSTValue}
	% require that the hashlock is the same
	\begin{umlcallself}[op={require(H(unlockSecret)=  hashLock)}]{OpenSTValue}
	\end{umlcallself}
	%expiraton height reqiure
	\begin{umlcallself}[op={require unstake expirationHeight > block number}]{OpenSTValue}
	\end{umlcallself}
   %call on simple stake from openSTValue to relase funds
	\begin{umlcall}[op={: releaseTo(beneficiary, amountST)}]{OpenSTValue}{SimpleStake}
	%release to transfers funds to Beneficiary
	\begin{umlcall}[op={: transfer(beneficiary, amountST)}]{SimpleStake}{OST}
	  \end{umlcall}
	%transfer the amount
   \begin{umlcall}[type=return, fill=green!20, op={<<OST(amountST)>>}]{SimpleStake}{OST}
      \begin{umlcall}[type=return, fill=green!20, op={<<OST(amountST)>>}]{OST}{Beneficiary}
        \end{umlcall} 
   %emit event Processed Unstake
   \begin{umlcall}[dt= 21, type=return, op={emit ProcessedUnstake(uuid, redemptionIntentHash, stakeAddress, redeemer, beneficiary, amountST, unlockSecret)}]{OpenSTValue}{OstDotCom}
  	 \end{umlcall}
  %\begin{umlcall}[type=return, op={emit ProcessedUnstake}]{OpenSTValue}{Beneficiary}
%\end{umlcall}
   %delete the storage for unstakes
   \begin{umlcallself}[op={delete unstakes [redemptionIntentHash]}]{OpenSTValue}
	\end{umlcallself}
  \end{umlcall}

  	\end{umlcall}
  \end{umlcall}
  
  \umlfpart[revert]
    	%revert unstaking could be initiatied from the redeemer side
	% beneficiary calls on processUnstaking with redempition intent hash
	\begin{umlcall}[dt= 10, op={: revertUnstaking(redemptionIntentHash)}]{OstDotCom}{OpenSTValue}
	% require that the hashlock is the same
	\begin{umlcallself}[op={require expirationHeight<= blockNumber}]{OpenSTValue}
	\end{umlcallself}
	%expiraton height reqiure
	%\begin{umlcallself}[op={require unstake expirationHeight > block number} ]{OpenSTValue}
	%\end{umlcallself}
		%call on simple stake from openSTValue to relase funds
	%\begin{umlcall}[op={: releaseTo(beneficiary, amountST)}]{OpenSTValue}{SimpleStake}
	   %delete the storage for unstakes
   \begin{umlcallself}[op={delete unstakes[redemptionIntentHash]}]{OpenSTValue}
	\end{umlcallself}
	%release to transfers funds to Beneficiary
	%\begin{umlcall}[dt=16, op={: transfer(beneficiary, amountST)}]{OpenSTValue}{SimpleStake}
	%transfer the amount
   %\begin{umlcall}[type=return, fill=green!20, op={<<OST(amountST)>>}]{SimpleStake}{Beneficiary}
   %emit event Processed Unstake
   \begin{umlcall}[type=return, dt= 10, op={emit RevertedUnstake(uuid, redemptionIntentHash, redeemer, beneficiary, amountST)}]{OpenSTValue}{OstDotCom}
  	 \end{umlcall}
    %\begin{umlcall}[type=return, dt= 10, op={emit RevertedUnstake}]{OpenSTValue}{OstDotCom}
  	 %\end{umlcall} 	 
     %\end{umlcall} 
  \end{umlcall}
  \end{umlfragment}
  
   \end{umlseqdiag}
\end{tikzpicture}
\end{document}

